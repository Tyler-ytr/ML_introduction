%TC第29.1节练习 4、5
%TC第29.2节练习 2、4、6
%TC第29.3节练习 5
%TC第29.4节练习 2
%%%%%%%%%%%%%%%%%%%%%%%%%%%%%%%%%%%%%%%%%%%%%%%%%%%%%%%%%%%%%%%%
\documentclass[11pt, a4paper, UTF8]{ctexart}
\input{preamble}

\title{4-1:线性规划 [课前]}
\me{张天昀}{171860508}
\date{\today}

\begin{document}
\maketitle
\noplagiarism

%%%%%%%%%%%%%%%%%%%%%%%%%%%%%%%%%%%%%%%%%%%%%%%%%%%%%%%%%%%%%%%%
%                       Homework START!                        %
%%%%%%%%%%%%%%%%%%%%%%%%%%%%%%%%%%%%%%%%%%%%%%%%%%%%%%%%%%%%%%%%
\beginthishw
%%%%%%%%%%%%%%%%%%%%
\begin{problem}[TC 29.1-4]
Convert the following linear program into standard form:\\
minimize
\begin{center}
\begin{tabular}{r r r r r r r}
$2x_1$& $+$ & $7x_2$ & $+$ & $x_3$ & & 
\end{tabular}
\end{center}
subject to
\begin{center}
\begin{tabular}{r r r r r r r}
$x_1$ & & & $-$ & $x_3$ & $=$ & 7\\
$3x_1$ & + & $x_2$ & & & $\geqslant$ & 24\\
 & & $x_2$ & & & $\geqslant$ & 0 \\
 & & & & $x_3$ & $\leqslant$ & 0 \\
\end{tabular}
\end{center}
\end{problem}
\begin{solution}
    
\end{solution}




\begin{problem}[TC 29.2-2]
Write out explicitly the linear program corresponding to finding the shortest path from node $s$ to node $y$ in Figure 24.2(a).
\begin{center}
    \includegraphics[scale=0.5]{24-2-a.png}
\end{center}
\end{problem}
\begin{solution}
    
\end{solution}




\begin{problem}[TC 29.2-4]
    Write out explicitly the linear program corresponding to finding the maximum flow
    in Figure 26.1(a).
\end{problem}
\begin{center}
\includegraphics[scale=0.5]{26-1-a.png}
\end{center}
\begin{solution}
    
\end{solution}




\begin{problem}[TC 29.2-6]
Write a linear program that, given a bipartite graph $G =(V,E)$, solves the maximum-bipartite-matching problem.
\end{problem}
\begin{solution}
    
\end{solution}





\begin{problem}[TC 29.3-5]
Solve the following linear program using \textsc{Simplex}:
maximize 
\begin{center}
\begin{tabular}{rrrrr}
$18x_1$ & $+$ & $12.5x_2$
\end{tabular}
\end{center}
subject to
\begin{center}
\begin{tabular}{rrrrr}
$x_1$ & $+$ & $x_2$ & $\leqslant$ & $20$ \\
$x_1$ & $ $ & $ $ & $\leqslant$ & $12$ \\
$ $ & $ $ & $x_2$ & $\leqslant$ & $16$ \\
$x_1,$ & $ $ & $x_2$ & $\geqslant$ & $0$ \\
%$ $ & $ $ & $ $ & $ $ & $ $ \\
\end{tabular}
\end{center}
\end{problem}
\begin{solution}
    
\end{solution}





\begin{problem}[TC 29.4-2]
Suppose that we have a linear program that is not in standard form. We could produce the dual by first converting it to standard form, and then taking the dual. It would be more convenient, however, to be able to produce the dual directly. Explain how we can directly take the dual of an arbitrary linear program.
\end{problem}
\begin{solution}
    
\end{solution}
%%%%%%%%%%%%%%%%%%%%%%%%%%%%%%%%%%%%%%%%%%%%%%%%%%%%%%%%%%%%%%%%
%                      Correction START!                       %
%%%%%%%%%%%%%%%%%%%%%%%%%%%%%%%%%%%%%%%%%%%%%%%%%%%%%%%%%%%%%%%%
%\begincorrection
%%%%%%%%%%%%%%%%%%%%
%\begin{problem}[]

%\end{problem}

%\begin{cause}
%
%\end{cause}

%\begin{revision}

%\end{revision}
%%%%%%%%%%%%%%%%%%%%
%\newpage
%%%%%%%%%%%%%%%%%%%%





%%%%%%%%%%%%%%%%%%%%%%%%%%%%%%%%%%%%%%%%%%%%%%%%%%%%%%%%%%%%%%%%
%                       Feedback START!                        %
%%%%%%%%%%%%%%%%%%%%%%%%%%%%%%%%%%%%%%%%%%%%%%%%%%%%%%%%%%%%%%%%
%\beginfb
%\begin{itemize}
%
%\end{itemize}





%%%%%%%%%%%%%%%%%%%%%%%%%%%%%%%%%%%%%%%%%%%%%%%%%%%%%%%%%%%%%%%%
%                        Homework END!                         %
%%%%%%%%%%%%%%%%%%%%%%%%%%%%%%%%%%%%%%%%%%%%%%%%%%%%%%%%%%%%%%%%
\end{document}
